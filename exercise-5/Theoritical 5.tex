% header
\documentclass[10pt,a4paper]{article}

\usepackage[latin1]{inputenc}
\usepackage{hyperref}
\usepackage{amssymb}
\usepackage{amsmath}

% the document
\begin{document}

% create the title
% Please replace the data in brackets [] with actual data.
\title{Solution - Exercise [$5$]\\
\small{Introduction to Computer Graphics - B-IT Master Course}}
\author{ [Melisa Cecilia] \and [Duy Khanh Gian] \and [Chenyu Zhao]}
\date{\today}
\maketitle

\section*{Theoretical Exercise}
{\bf Clipping algorithms } \\

This exercise refers to the "raster algorithms" slideset of the lecture. \\

\begin{flushleft}
1. Describe what changes are necessary to generalize the Liang-Barsky line clipping algorithm to n-dimensional space
\end{flushleft}

 We will need to use the same logic for 2D clipping (comparing the line with 4 intersection points, 2 for x and 2 for y) with the 2*n intersection points in n-dimensional space. The number 2 is because each line inside the volume has a maximum of 2 intersections (max and min). So for 3D space, with Liang-Barsky we will need to compare the lines with 6 intersection points (2 for x, 2 for y, and 2 for z) \\
 
\begin{flushleft}
2. Visualize the logic of the Sutherland-Hodgman polygon clipping algorithm using either a flow diagram or pseudo code
\end{flushleft}

\begin{align*}
  &\text{For Each Edge in clipping polygon} \\
  & \hspace{1cm} \text{create plane for edge} \\
  & \hspace{1cm} \text{For Each Vertex pair in subject polygon} \\
  	& \hspace{2cm} \text{determine case number from starting vertex, ending vertex, and edge} \\
	& \hspace{2cm} \text{perform case operation} \\
  & \hspace{1cm} \text{End For Each} \\
  & \hspace{1cm} \text{update subject polygon with new vertices} \\
  &\text{End For Each}
\end{align*}

\end{document}