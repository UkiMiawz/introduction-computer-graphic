% header
\documentclass[10pt,a4paper]{article}

\usepackage[latin1]{inputenc}
\usepackage{hyperref}
\usepackage{amssymb}
\usepackage{amsmath}

% the document
\begin{document}

% create the title
% Please replace the data in brackets [] with actual data.
\title{Solution - Exercise [$2$]\\
\small{Introduction to Computer Graphics - B-IT Master Course}}
\author{ [Melisa Cecilia] \and [Duy Khanh Gian] \and [Chenyu Zhao]}
\date{\today}
\maketitle

\section*{Exercise 1}
Given are two points p1,p2 on the unit sphere in ?3. Using a quaternion the point p1 is to be rotated onto the point p2.\\

a. Give a formula to determine the angle of rotation $\alpha$
\begin{gather*}
\alpha = \arccos{\{\dfrac{q_1.q_2}{|q_1|.|q_2|}\}}
\end{gather*}

b. Give a formula to detemine the rotation axis v
\begin{gather*}
v = q_1.q_2
\end{gather*}

c. Write down the quaternion q which performs the rotation with angle $\alpha$ around v

\begin{gather*}
a = rotation angle \\
x,y,z = rotation axis \\
q = \cos(\dfrac{a}{2}) + i (x \sin\dfrac{a}{2}) + j (y \sin\dfrac{a}{2}) + k ( z \sin\dfrac{a}{2})\\
\end{gather*}

d. Write down the relationship between p1 and p2 using quaternion multiplication
\begin{gather*}
q_1 = q_{10} + \textbf{i} q_{11} + \textbf{j} q_{12} + \textbf{k} q_{13} \\
q_2 = q_{20} + \textbf{i} q_{21} + \textbf{j} q_{22} + \textbf{k} q_{23}
\end{gather*}

\begin{multline*}
q_1 � q_2 = (q_{10}q_{20} - q_{11}q_{21} - q_{13}q_{23} - q_{14}) + \textbf{i} (q_{11}q_{20} + q_{10}q_{21} + q_{13}q_{23} - q_{14}q_{22}) \\ 
+ \textbf{j} (q_{10}q_{22} - q_{11}q_{23} + q_{12}q_{20} + q_{13}q_{21} ) + \textbf{k} (q_{10}q_{23} + q_{11}q_{22} - q_{12}q_{21} + q_{13}q_{20})
\end{multline*}

\section*{Exercise 2}

Given a point p $\epsilon$ $\mathbb{R}$ 3 in homogenous coordinates (x y z 1)\textsuperscript{T}: \\

Rotation Matrix \\ \\
\(
R_z=
  \begin{pmatrix}
    \cos(\alpha) & -\sin(\alpha) & 0 & 0 \\
    \sin(\alpha) & \cos(\alpha) & 0 & 0\\
    0 & 0 & 1 & 0 \\
    0 & 0 & 0 & 1
  \end{pmatrix}
\) \\ \\

Translation Matrix \\ \\
\(
T_t=
  \begin{pmatrix}
    1 & 0 & 0 & t_1 \\
    0 & 1 & 0 & t_2\\
    0 & 0 & 1 & t_3 \\
    0 & 0 & 0 & 1
  \end{pmatrix}
\) \\ \\

a. Derive a matrix $M_1$ which first rotates the point $\alpha$ degrees ($\alpha$ is given in radians) around the axis (0 0 1)T and then performs a translation with an offset of (t1 t2 t3)\textsuperscript{T} \\

\begin{equation*} \label{eq_M1}
\begin{aligned}
M_1 &= 
  \begin{pmatrix}
    \cos(\alpha) & -\sin(\alpha) & 0 & 0 \\
    \sin(\alpha) & \cos(\alpha) & 0 & 0\\
    0 & 0 & 1 & 0 \\
    0 & 0 & 0 & 1
  \end{pmatrix} . 
    \begin{pmatrix}
    1 & 0 & 0 & t_1 \\
    0 & 1 & 0 & t_2\\
    0 & 0 & 1 & t_3 \\
    0 & 0 & 0 & 1
  \end{pmatrix} \\ \\
  &= 
  \begin{pmatrix}
    \cos(\alpha) & -\sin(\alpha) & 0 & t_1\cos(\alpha) - t_2\sin(\alpha) \\
    \sin(\alpha) & \cos(\alpha) & 0 & t_1\sin(\alpha) + t_2\cos(\alpha) \\
    0 & 0 & 1 & t_3 \\
    0 & 0 & 0 & 1
  \end{pmatrix}
\end{aligned}
\end{equation*} \\ \\

b. Derive a matrix $M_2$ which first performs a translation with an offset of (t1 t2 t3)\textsuperscript{T}  and then rotates the point $\alpha$ degrees ($\alpha$ is given in radians) around the axis (0 0 1)\textsuperscript{T} \\
s\begin{equation*} \label{eq_M1}
\begin{aligned}
M_2 &= 
    \begin{pmatrix}
    1 & 0 & 0 & t_1 \\
    0 & 1 & 0 & t_2\\
    0 & 0 & 1 & t_3 \\
    0 & 0 & 0 & 1
  \end{pmatrix} . 
  \begin{pmatrix}
    \cos(\alpha) & -\sin(\alpha) & 0 & 0 \\
    \sin(\alpha) & \cos(\alpha) & 0 & 0\\
    0 & 0 & 1 & 0 \\
    0 & 0 & 0 & 1
  \end{pmatrix} \\ \\
  &= 
  \begin{pmatrix}
    \cos(\alpha) & -\sin(\alpha) & 0 & t_1 \\
    \sin(\alpha) & \cos(\alpha) & 0 & t_2\\
    0 & 0 & 1 & t_3 \\
    0 & 0 & 0 & 1
  \end{pmatrix}
\end{aligned}
\end{equation*}

\end{document}